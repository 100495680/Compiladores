\documentclass[12pt,a4paper]{article}
\usepackage[utf8]{inputenc}
\usepackage[T1]{fontenc}
\usepackage[spanish]{babel}
\usepackage{graphicx}
\usepackage{listings}
\usepackage{xcolor}
\lstset{
    basicstyle=\ttfamily\small,
    keywordstyle=\color{blue}\bfseries,
    commentstyle=\color{gray}\itshape,
    stringstyle=\color{red},
    breaklines=true,
    numbers=left,
    numberstyle=\tiny,
    numbersep=5pt,
    frame=single,
    captionpos=b
}

\begin{document}

% Portada
\begin{titlepage}
    \centering
    \vspace*{2cm}
    {\LARGE\textbf{PROCESADORES DEL LENGUAJE}\par}
    \vspace{1cm}
    {\Large Práctica Final: Frontend y Backend\par}
    \vfill
    \includegraphics[width=0.65\textwidth]{logo_universidad.png}
    \vfill
    {\large\textbf{Grupo:} 113-leal-zhu\par}
    {\large\textbf{ID de grupo de prácticas:} 81\par}
    \vspace{1cm}
    {\large\textbf{Nombre de los participantes:}\par}
    {\normalsize Liang Ji Zhu \\ Ignacio Leal Sánchez\par}
    {\large\textbf{Correos electrónicos de los participantes:}\par}
    {\normalsize 100495723@alumnos.uc3m.es \\ 100495680@alumnos.uc3m.es\par}
    \vfill
    {\large Madrid, \today\par}
\end{titlepage}

% Índice automático según estructura de secciones
\clearpage
\tableofcontents
\clearpage

% Introducción
\section{Introducción}
\noindent La presente memoria documenta el desarrollo integral de un traductor frontend y un traductor backend. 
El trabajo plantea la construcción de éstas en dos fases:
\begin{itemize}
    \item Frontend: traduce el lenguaje C, de notación infija, a código Lisp, de notación prefija.
    \item Backend: convierte dicho código Lisp a notación postfija, Forth, siguiendo la filosofía de máquina de pila integrada en gforth.
\end{itemize}

\noindent Se sifuió un ciclo incremental para la realización del proyecto, en cada iteración se añadió una nueva producción. 
Esta estrategia permitió aislar errores tempranos y dosificar la complejidad, ya que los apartados iniciales 
eran más asequibles y los últimos exigían mayor autonomía y un diseño gramatical más depurado.

\smallskip
\noindent Se puede apreciar 3 etapas de implementación aproximadas:
\begin{itemize}
    \item Etapa 1: se implementaron las declaraciones básicas, la función \texttt{main} y las primeras 
    sentencias de E/S (\texttt{puts}, \texttt{printf}).
    \item Etapa 2: se integraron operadores con su precedencia y las estructuras de control 
    \texttt{while} e \texttt{if/else}.
    \item Etapa 3: se añadieron funciones con retorno estructurado y vectores.
    \item Etapa 4: desarrollo del backend y pruebas funcionales con la directiva 
    \texttt{//@ (main)}, completando el flujo C $\to$ Lisp $\to$ Forth.
\end{itemize}

\smallskip
\noindent El analizador léxico se mantuvo sin modificaciones; toda la lógica nueva se centró en la 
gramática y las acciones semánticas, añadiendon los tokens necesarios.
\smallskip

\noindent Cada incremento semanal concluyó con una regresión completa sobre el conjunto de casos de prueba 
—tanto los oficiales como los diseñados ad-hoc— y con la documentación de las decisiones en esta memoria. 
Así, el resultado final es un traductor robusto cuya construcción refleja paso a paso la ruta marcada por 
las pautas del enunciado, desde las bases del frontend hasta los detalles finales del backend.
% Otras secciones según índice
\section{Transparencia}

\noindent En este apartado constatamos que el uso de la IA generativa ha sido utilizado para la 
corrección de errores en las declaraciones de C, aunque ha sido intercalado con la asistencia de los 
profesores que proponen respuestas más coherentes y simples. Y ha sido imposible usar la inteligencia 
artificial en más partes del trabajo debido a su incapacidad para comprender los requisitos específicos 
del problema, y por ello dando respuestas de nula utilidad.

\smallskip
\noindent Asimismo, hacemos constar que la carga de trabajo se ha repartido de forma justa entre los dos. 
A lo largo de todo el período —y, muy especialmente, durante la fase final— la dedicación fue dinámica: 
cada miembro asumió más tareas cuando su disponibilidad lo permitía y cedió responsabilidad cuando sus 
circunstancias lo exigían, garantizando en todo momento un equilibrio global.

\section{Traducción Frontend}
\subsection{Variables Globales}
\noindent Las variables se declaran como enteros o como arrays en los que se puede dar valor o no, en este caso se 
usará el valor 0 para la traducción a lisp. Se permite que se declaren múltiples variables en línea 
separadas con "," o en cascada.
\begin{lstlisting}
axioma:             var_global funcion
                    r_axioma
                    ;
r_axioma:
                    |   axioma
                    ;

var_global:         declaracion ';'  var_global
                    |
                    ;

declaracion:        INTEGER  IDENTIF valor_global r_declaracion
                    | INTEGER  IDENTIF '[' NUMBER ']' r_declaracion
                    ;

valor_global:
                    | '=' NUMBER
                    ;
r_declaracion:      ',' IDENTIF valor_global r_declaracion
                    | ',' IDENTIF '[' NUMBER ']' r_declaracion
                    |
                    ;
\end{lstlisting}
\subsection{Función MAIN, Funciones Genéricas y Variables Locales}

\noindent Se definen las funciones de dos maneras: las funciones genéricas y el main. El main es una función 
diferente ya que es obligatorio que exista esta función. 
\noindent Dentro del cuerpo de las funciones pueden declararse variables,
que tanto en C como en Lisp serán variables locales. Se emplea la misma traducción
de las globales, pero en este caso deben generarse dentro del cuerpo de la función.
Por lo que también permitimos la declaración de variables locales que serán la variable 
seguida del nombre de la función en la que es variable local.
\begin{lstlisting}
funcion:            IDENTIF  '(' argumento ')' '{' var_local cuerpo '}' funcion  
                    | funcion_principal                                            
                    ;

funcion_principal:  MAIN  '(' argumento ')' '{' var_local cuerpo '}'               
                    ;

argumento:          INTEGER valor resto_argumento                    
                    | valor resto_argumento                       
                    |                                               
                    ;

valor:              STRING                                           
                    | expresion                                      
                    ;

resto_argumento:    ',' argumento                                 
                    |                                              
                    ;   


var_local:          declaracion_local ';' var_local               
                    |                                              
                    ;

declaracion_local:  INTEGER  IDENTIF valor_local r_decl_local       
                    | INTEGER  IDENTIF '[' NUMBER ']' r_decl_local  
                    ;

valor_local:        /* lambda */                                    
                    | '=' NUMBER                                    
                    ;
r_decl_local:       ',' IDENTIF valor_local r_decl_local            
                    | ',' IDENTIF '[' NUMBER ']' r_decl_local        
                    |                                              
                    ;
\end{lstlisting}
\subsection{Impresión de Cadenas}
\noindent La impresión de cadenas de texto se hace imprimiendo o un string o una variable. No existe 
la forma de impresión formateada como en C. Por ello cualquier impresión de variables en C se 
traducirá a la impresión de variables individuales en Lisp.
\begin{lstlisting}
sentencia:          asignacion                                   
                    | '@' expresion                              
                    | PUTS '(' STRING ')'                       
                    | PRINTF printf                            
                    | llamada                                  
                    ;
printf:             '(' STRING r_printf ')'                    
                    ;
r_printf:           ',' expresion r_printf                    
                    | ',' STRING r_printf                      
                    |                                           
                    ;
\end{lstlisting}
\subsection{Operadores, Precedencia, Asociatividad y Vectores}
\noindent Aquí hemos cambiado la forma de operar respecto a la versión anterior. Para garantizar 
la mejor precedencia sin ninguna posibilidad de errores hemos dividido las diferentes operaciones 
o comparaciones en No Terminales específicos de forma que la precedencia pueda ser cumplida en todo 
momento.
\noindent Para declarar un Vector podemos usar una sintaxis de Lisp extendiendo la que usamos para 
variables y podrán declararse como variables globales o locales. 
\begin{lstlisting}
var_global:         declaracion ';'  var_global                         
                    |                                                   
                    ;

declaracion:        INTEGER  IDENTIF valor_global r_declaracion         
                    | INTEGER  IDENTIF '[' NUMBER ']' r_declaracion
                    // Vector Global     
                    ; 
...
/* ==== Operadores, precedencia y asociatividad ==== */
asignacion:         IDENTIF '=' expresion                               
                    | vector '=' expresion    
                    // Vector Local                           
                    ;

expresion:          logical_or                                         
                    ;

llamada:            IDENTIF '(' argumento ')'                           
                    ;

logical_or:         logical_and                                         
                    | logical_or OR logical_and                         
                    ;
logical_and:        igualdad                                            
                    | logical_and AND igualdad                          
                    ;
igualdad:           relacional                                          
                    | igualdad EQ relacional                            
                    | igualdad NE relacional                            
                    ;
relacional:         aditivo                                             
                    | relacional '<' aditivo                            
                    | relacional '>' aditivo                            
                    | relacional LE aditivo                             
                    | relacional GE aditivo                             
                    ;
aditivo:            multiplicativo                                      
                    | aditivo '+' multiplicativo                        
                    | aditivo '-' multiplicativo                        
                    ;
multiplicativo:     unario                                              
                    | multiplicativo '*' unario                        
                    | multiplicativo '/' unario                         
                    | multiplicativo '%' unario                         
                    ;
unario:             operando                                            
                    | '!' unario                                        
                    | '+' operando %prec UNARY_SIGN                     
                    | '-' operando %prec UNARY_SIGN                      
                    ;

operando:           IDENTIF                                            
                    | IDENTIF '(' argumento ')'                         
                    | NUMBER                                            
                    | '(' logical_or ')'                                
                    |  vector                                           
                    ;

vector:             IDENTIF '[' logical_or ']'
                    ;
\end{lstlisting}
\noindent Se declararon \textit{tokens} para los operadores lógicos y de comparación con el objetivo 
de mejorar la legibilidad del código y establecer un acoplamiento más claro entre las fases de 
análisis léxico y análisis sintáctico.

\noindent En particular, los operadores de comparación fueron definidos como \texttt{nonassoc}, 
ya que, a diferencia de operadores como la suma o la multiplicación, las comparaciones 
(por ejemplo, \texttt{==}, \texttt{!=}) no son asociativas. Esto significa que una expresión como:
\begin{center}
\texttt{a == b == c}
\end{center}
\noindent
no tiene un significado bien definido en C. Existen dos posibles formas de interpretarla:

\begin{center}
\texttt{(a == b) == c} || \texttt{a == (b == c)}
\end{center}

\noindent Ambas interpretaciones pueden generar ambigüedad semántica. Al declarar estos operadores 
como \texttt{nonassoc}, el \textit{parser} rechaza expresiones como \texttt{a == b == c} y obliga 
al programador a escribirlas de manera explícita, por ejemplo:

\begin{itemize}
    \item \texttt{(a == b) \&\& (b == c)}
    \item Uso de paréntesis para aclarar el orden de evaluación
\end{itemize}

\noindent Este enfoque permite capturar errores de lógica o errores tipográficos en tiempo de 
compilación, evitando comportamientos ambiguos o inesperados en tiempo de ejecución.
\begin{lstlisting}
...
%token EQ
%token NE
%token LE
%token GE
%token OR
%token AND

%right '='                              
%left OR                               
%left AND                               
%nonassoc EQ NE                         
%nonassoc '<' '>' LE GE                 
%left '+' '-'                           
%left '*' '/' '%'                       
%right UNARY_SIGN "!"                   
\end{lstlisting}
\subsection{Estructuras de Control}
\noindent Las estructuras de control se encuentran dentro del cuerpo de una función y no acaban en ";". 
Estas son el IF-ELSE, FOR y WHILE. Cuentan con una estructura de cuerpo diferente ya que un return de 
alguna de estas estructuras se interpreta diferente ya que no es un return al final de la función.
\begin{lstlisting}
cuerpo:             sentencia ';' cuerpo                        
                    | sentencia ';'                                
                    | estructura cuerpo                                                                    
                    | estructura                                  
                    | RETURN expresion ';'                        
                    ;
estructura:         WHILE '(' expresion ')' '{' cuerpo_estructura '}'                                    
                    | IF '(' expresion ')' '{' cuerpo_estructura '}'                                     
                    | IF '(' expresion ')' '{' cuerpo_estructura '}' ELSE '{' cuerpo_estructura '}'                                                                        
                    | FOR '(' declaracion_for ';' expresion ';' asignacion ')' '{' cuerpo_estructura '}'                                        
                    ;

declaracion_for:    INTEGER  IDENTIF valor_for r_declaracion_for    
                    |      IDENTIF valor_for r_declaracion_for                                           
                    ;
valor_for:                                                          
                    | '=' NUMBER                                                                                      
                    ;
r_declaracion_for:      ',' IDENTIF valor_for r_declaracion_for                                                    
                    |                                              
                    ;
cuerpo_estructura:  sentencia ';'                               
                    | estructura                                      
                    | sentencia ';' cuerpo_estructura           
                    | estructura cuerpo_estructura                        
                    | RETURN expresion ';'                                                           
                    ;
\end{lstlisting}

\section{Traducción Backend}
\noindent Una vez acabado con la implementación del frontend que traducía de C a Lisp, vamos a 
realizar el correspondiente backend, de Lisp a Forth. Para ello, hemos decidido dividir la 
implementación en las siguientes fases:
\subsection{Variables Globales}
\noindent Tal y como traducimos el lisp todas las variables que pueden haber en este código serán 
globales. Simplemente cambiará el nombre si son específicas a una función en específico. Como ya 
se unificaron las formas de declarar variables esta parte de la gramática es relativamente simple.
\begin{lstlisting}
var_global:         declaracion                                                        
                    | var_global declaracion
                    ;                               
declaracion:        '(' SETQ IDENTIF logical_or ')'
                    ; 
\end{lstlisting}
\subsection{Función MAIN y Funciones Genéricas}
\noindent La definición de funciones se puede unificar para que haya un no terminal que acoge ambos 
dos tipos de posibles funciones el main que sigue siendo obligatorio y las otras que siguen siendo 
opcionales. Esta es una forma más simple que la utilizada en el Frontend y por la traducción de Lisp 
no se nos requerirá el uso de argumentos en estas funciones.
\begin{lstlisting}
def_funcs:          def_funcs def_func                
                    | def_func                                                         
                    | def_funcs llamada_main       
                    | llamada_main                                                      
                    ;
llamada_main:       '(' MAIN ')'

def_func:           '(' DEFUN MAIN '(' ')' cuerpo ')'
                    | '(' DEFUN IDENTIF '(' ')' cuerpo ')'
                    ;  
\end{lstlisting}
\subsection{Impresión de Cadenas}
\noindent Esta función ya fue muy alterada para el cambio de C a Lisp y aquí la traducción es 
bastante simple ya que el programa puede necesitar imprimir strings o variables/resultados de 
operaciones. 
\begin{lstlisting}
sentencia:          '(' PRINT STRING ')'            
                    | '(' PRINC logical_or ')'          
                    | '(' PRINC STRING ')' 
                    ...
                    ;
\end{lstlisting}
\subsection{Operadores, Precedencia y Asociatividad}
\noindent Estos son todos los operadores y comparadores que permiten que se sigan las reglas de 
asociatividad que creamos en la primera gramática. Esta parte de la gramática es muy similar a la 
del Frontend ya que el cambio de lenguaje no realiza ningún cambio estructural muy grande. Lo que sí
que tuvimos en cuenta fue en cuanto a las operaciones en Lisp, que al llevar paréntesis hicimos que las
operaciones tuvieran el mismo nivel de precedencia.
\begin{lstlisting}
logical_or:         logical_and                                                         
                    | '(' OR logical_or logical_and ')'      
                    ;                               
logical_and:        igualdad                                                            
                    | '(' AND logical_and igualdad ')'       
                    ;                               
igualdad:           relacional                                                          
                    | '(' '=' igualdad relacional ')'        
                    | '(' NE igualdad relacional ')'        
                    ;                               
relacional:         operacion                                                          
                    | '(' '<' relacional operacion ')'     
                    | '(' '>' relacional operacion ')'   
                    | '(' LE relacional operacion ')'   
                    | '(' GE relacional operacion ')'    
                    ;                               
operacion:          unario                                                              
                    | '(' '+' operacion operacion ')'    
                    | '(' '-' operacion operacion ')'  
                    | '(' '*' operacion operacion ')'  
                    | '(' '/' operacion operacion ')'  
                    | '(' MOD operacion operacion ')'  
                    ;                                                   
unario:             operando                                                            
                    | '(' NOT unario ')'                           
                    | '+' operando %prec UNARY_SIGN                                    
                    | '(' '-' operando %prec UNARY_SIGN ')'   
                    ;
operando:           IDENTIF
                    | NUMBER                                                            
                    | '(' logical_or ')'                                                
                    ;
\end{lstlisting}
\subsection{Estructuras de Control}
\noindent Hemos logrado que en esta gramática las sentencias y estructuras de control estén 
unificadas gracias a la estructura de lisp y a la forma tan estandarizada de definir el código.
\begin{lstlisting}
sentencia:          '(' PRINT STRING ')'            
                    | '(' PRINC logical_or ')'          
                    | '(' PRINC STRING ')'                 
                    | '(' SETF IDENTIF logical_or ')'           
                    | '(' SETQ IDENTIF logical_or ')'               
                    | '(' LOOP WHILE logical_or DO lista_sentencia ')'   
                    | '(' IF logical_or sentencia ')'         
                    | '(' IF logical_or sentencia sentencia ')'     
                    | '(' PROGN lista_sentencia ')' 
                    ;
\end{lstlisting}

\section{Testing}
\noindent En esta sección se presentan las pruebas realizadas para verificar el 
correcto funcionamiento del traductor en sus distintas etapas. Cada prueba incluye 
el código fuente en C, su correspondiente traducción a Lisp y finalmente la 
traducción a Forth.
\subsection{Variables Globales}
\begin{lstlisting}
#include <stdio.h>
int a;
int b = 10, c, d = 5;
main() {
    puts("Variables globales");
}
//@ (main)
---------------------------------------------------
Variables globales
---------------------------------------------------
(setq a 0)
(setq b 10)
(setq c 0)
(setq d 5)
(defun main ()
	(print "Variables globales")
)
 (main)
---------------------------------------------------
"Variables globales" 
---------------------------------------------------
variable a
0 a !
variable b
10 b !
variable c
0 c !
variable d
5 d !
: main ." Variables globales" ;
main
---------------------------------------------------
Variables globales
\end{lstlisting}
\subsection{Función MAIN}
\begin{lstlisting}
#include <stdio.h>
int x;

main() {
    x = x + 1;
    printf("%d\n", x);
}
//@ (main)
---------------------------------------------------
1
---------------------------------------------------
(setq x 0)
(defun main ()
	(setf x (+ x 1))
	(princ x)
	
)
 (main)
---------------------------------------------------
1
---------------------------------------------------
variable x
0 x !
: main x @ 1 + x !
x @ . ;
main
---------------------------------------------------
1 
\end{lstlisting}
\subsection{PUTS}
\begin{lstlisting}
#include <stdio.h>
int main_var;
main() {
    puts("Hola, mundo!");
}
//@ (main)
---------------------------------------------------
Hola, mundo!
---------------------------------------------------
(setq main_var 0)
(defun main ()
	(print "Hola, mundo!")
)
 (main)
---------------------------------------------------
"Hola, mundo!" 
---------------------------------------------------
variable main_var
0 main_var !
: main ." Hola, mundo!" ;
main
---------------------------------------------------
Hola, mundo!
\end{lstlisting}
\subsection{PRINT Simple}
\begin{lstlisting}
#include <stdio.h>
int val = 42;
main() {
    printf("%d", val);
}
//@ (main)
---------------------------------------------------
42
---------------------------------------------------
(setq val 42)
(defun main ()
	(princ val)
	
)
 (main)
---------------------------------------------------
42
---------------------------------------------------
variable val
42 val !
: main val @ . ;
main
---------------------------------------------------
42 
\end{lstlisting}
\subsection{Operadores}
\begin{lstlisting}
#include <stdio.h>
int a = 1;
int b = 0;
int c = 2;
main() {
    // mezcla &&, ||, ==, !=, <, >, <=, >=, %, +, -, *, /
    if ((a + c * 3) % 2 == 1 && b != 0 || a <= c) {
        puts("OK");
    }
}
//@ (main)
---------------------------------------------------
OK
---------------------------------------------------
(setq a 1)
(setq b 0)
(setq c 2)
(defun main ()
	(if (or (and (= (mod (+ a (* c 3)) 2) 1) (/= b 0)) (<= a c))
	(progn	(print "OK")))
)
 (main)
---------------------------------------------------
"OK" 
---------------------------------------------------
variable a
1 a !
variable b
0 b !
variable c
2 c !
: main a @ c @ 3 * + 2 mod 1 = b @ 0 = 0= and a @ c @ <= or if 
	." OK" 
then ;
main
---------------------------------------------------
OK
\end{lstlisting}
\subsection{WHILE}
\begin{lstlisting}
#include <stdio.h>
int cnt = 0;
main() {
    while (cnt < 3) {
        printf("%d\n", cnt);
        cnt = cnt + 1;
    }
}
//@ (main)
---------------------------------------------------
0
1
2
---------------------------------------------------
(setq cnt 0)
(defun main ()
	(loop while (< cnt 3) do
	(progn	(princ cnt)
	
	(setf cnt (+ cnt 1))))
)
 (main)
---------------------------------------------------
012
---------------------------------------------------
variable cnt
0 cnt !
: main begin
	cnt @ 3 <
while
	cnt @ .
cnt @ 1 + cnt !
repeat ;
main
---------------------------------------------------
0 1 2 
\end{lstlisting}
\subsection{IF-ELSE}
\begin{lstlisting}
#include <stdio.h>
int v = 5;
int res;
main() {
    if (v % 2 == 0) {
        puts("Par");
    } else {
        puts("Impar");
    }
    if (v > 10) {
        puts("Grande");
    }
}
//@ (main)
---------------------------------------------------
Impar
---------------------------------------------------
(setq v 5)
(setq res 0)
(defun main ()
	(if (= (mod v 2) 0)
	(progn	(print "Par"))
	(progn	(print "Impar")))
	(if (> v 10)
	(progn	(print "Grande")))
)
 (main)
---------------------------------------------------
"Impar" 
---------------------------------------------------
variable v
5 v !
variable res
0 res !
: main v @ 2 mod 0 = if 
	." Par" 
else 
	." Impar" 
then
v @ 10 > if 
	." Grande" 
then ;
main
---------------------------------------------------
Impar
\end{lstlisting}
\subsection{FOR}
\begin{lstlisting}
#include <stdio.h>
int i;
int suma = 0;
int n = 5;
main() {

    for (i = 0; i < n; i = i + 1) {
        suma = suma + i;
    }
    printf("%d\n", suma);
}
//@ (main)
---------------------------------------------------
10
---------------------------------------------------
(setq i 0)
(setq suma 0)
(setq n 5)
(defun main ()
	(setq i 0)
	(loop while (< i n) do
	(progn	(setf suma (+ suma i)))
	(setf i (+ i 1)))
	(princ suma)
	
)
 (main)
---------------------------------------------------
10
---------------------------------------------------
variable i
0 i !
variable suma
0 suma !
variable n
5 n !
: main 0 i !
begin
	i @ n @ <
while
	suma @ i @ + suma !
i @ 1 + i !
repeat
suma @ . ;
main
---------------------------------------------------
10 
\end{lstlisting}
\subsection{Funciones (Lisp)}
\begin{lstlisting}
#include <stdio.h>
square (int v) {
    return v * v; 
    }
sum (int x, int y) { return x + y; }
main() {
    int r1, r2;
    r1 = square(7);
    r2 = sum(3, 4);
    printf("%d %d\n", r1, r2);
    // llamada como procedimiento (valor ignorado)
    square(5);
}
//@ (main)
---------------------------------------------------
49 7
---------------------------------------------------
(defun square (v )
	(* v v)
)
(defun sum (x  y )
	(+ x y)
)
	
(defun main ()
	(setq r1 0)
	(setq r2 0)
	(setf r1 (square 7 ))
	(setf r2 (sum 3  4 ))
	(princ r1)
	(princ r2)
	
	(square 5 )
)
 (main)
---------------------------------------------------
497
\end{lstlisting}
\subsection{Variables Locales (Lisp)}
\begin{lstlisting}
#include <stdio.h>
main() {
    int g = 1;
    int a = 4, b;
    a = a + 1;
    b = a * 2;
    printf("%d %d\n", a, b);
}
//@ (main)
---------------------------------------------------
5 10
---------------------------------------------------
(defun main ()
	(setq g 1)
	(setq a 4)
	(setq b 0)
	(setf a (+ a 1))
	(setf b (* a 2))
	(princ a)
	(princ b)
)
 (main)
---------------------------------------------------
510
\end{lstlisting}
\subsection{Vectores (Lisp)}
\begin{lstlisting}
#include <stdio.h>
int arr[5];
main() {
    int i;
    for (i = 0; i < 5; i = i + 1) {
        arr[i] = i * 2;
    }
    printf("%d %d %d %d %d\n",
           arr[0], arr[1], arr[2], arr[3], arr[4]);
}
//@ (main)
---------------------------------------------------
0 2 4 6 8
---------------------------------------------------
(setq arr (make-array 5))
(defun main ()
	(setq i 0)
	(setq i 0)
	(loop while (< i 5) do
	(progn	(setf (aref arr i) (* i 2)))
	(setf i (+ i 1)))
	(princ (aref arr 0))
	(princ (aref arr 1))
	(princ (aref arr 2))
	(princ (aref arr 3))
	(princ (aref arr 4))
)
 (main)
---------------------------------------------------
02468
\end{lstlisting}
\clearpage
\section{Conclusiones}

\noindent La presente memoria ha descrito en detalle el diseño, la implementación y la validación de un traductor 
de dos fases para un subconjunto de C: un \emph{frontend} que convierte C en Lisp y un \emph{backend} que 
convierte Lisp en Forth. A lo largo del proyecto se han abordado todos los objetivos formativos propuestos 
en la asignatura: la definición de gramáticas jerárquicas, la gestión de la semántica mediante código 
diferido, la generación de código intermedio y final, y la validación con conjuntos de pruebas 
diseñados ad hoc y suministrados por el profesorado.

\noindent El ciclo incremental de desarrollo permitió aislar y corregir errores de forma temprana en cada iteración. 
Gracias a esta estrategia, pudimos empezar por las declaraciones básicas y la impresión de cadenas, añadir luego 
operadores y estructuras de control, y finalmente incorporar funciones con retorno y vectores antes de abordar el 
backend. La generación de los ficheros \texttt{trad.y} y \texttt{back.y}, así como el conjunto de scripts de prueba 
automatizados, garantizó la robustez del traductor y facilitó la comparación de los comportamientos nativos en C, 
Lisp y Forth.

\noindent Como principales retos formativos cabe destacar la resolución de conflictos de gramática (shift/reduce y 
reduce/reduce), la correcta gestión de ámbitos en variables locales versus globales, y la inserción de instrucciones 
de salida (\texttt{bye} en GForth) para evitar bloqueos en los scripts automatizados. Estos desafíos han reforzado 
el entendimiento de la teoría de compiladores y la importancia de un buen diseño gramatical y semántico.

\noindent Para trabajos futuros, sería interesante ampliar el subconjunto de C soportado (por ejemplo, 
punteros, estructuras y llamadas con múltiples argumentos) o una implementación propio de un analizador 
léxico en \emph{flex}.

\noindent En definitiva, este proyecto no solo cumple los requisitos de la práctica, sino que sienta las bases para 
desarrollos posteriores en el ámbito de los procesadores de lenguaje y la generación de compiladores de 
propósito educativo o industrial.

\end{document}
